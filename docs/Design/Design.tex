\documentclass[11pt,fleqn]{article}

\usepackage{blindtext}
\usepackage{hyperref}
\usepackage{graphicx}
\graphicspath{ {./} }
\hypersetup{
    colorlinks=true,
    linkcolor=blue,
    filecolor=magenta,      
    urlcolor=cyan,
    pdftitle={Overleaf Example},
    pdfpagemode=FullScreen,
    }
\urlstyle{same}

\newcommand{\indentpar}{\phantom{=}}
\newcommand{\bli}{\begin{itemize}}
\newcommand{\eli}{\end{itemize}}

\begin{document}

    %% Title Page %%%%%%%%%%%%%%%%%%%%%%%%%%%%%%%%%%%%%%%%%%%%%%%%%
      
      \title{Design of VDisp Software}
      \date{\today}
      \author{Emil Soleymani, Dr.~Spencer Smith\\ McMaster University}
      \maketitle

      \medskip

      % Description
      \indentpar The \textbf{VDisp} software was carefully designed to adhere to the \href{https://www.digitalocean.com/community/conceptual_articles/s-o-l-i-d-the-first-five-principles-of-object-oriented-design}{\emph{SOLID} software design principles}.
      This document aims to outline the initial proposed design (which will be called the ideal design), the actual implemented
      design, and the limitations of the Julia programming language that lead to this drastic change.
      
    %%%%%%%%%%%%%%%%%%%%%%%%%%%%%%%%%%%%%%%%%%%%%%%%%%%%%%%%%%%%%%%
    
    \pagebreak

    %% Ideal Design %%%%%%%%%%%%%%%%%%%%%%%%%%%%%%%%%%%%%%%%%%%%%%%
    \section*{Ideal Design}

    \indentpar This section aims to describe the \emph{ideal} design which was drafted
    for the \textbf{VDisp} software. \\

    \begin{figure}[h]
        \includegraphics[width=1.3\textwidth]{VDispIdealDesignUML.png}
        \centering
    \end{figure}

    \indentpar The diagram above has been condensed. \emph{InputData} uses many 
    Custom Exception classes which help identify specific problems in the input file
    format allowing for helpful and descriptive messages to be given to the user. Furthermore,
    the \textbf{FoundationOutputBehaviour}, \textbf{DisplacementInfoBehaviour}, \textbf{ForcePointBehaviour} and
    \textbf{EquilibriumInfoBehaviour} interfaces (and the classes that implement them) have 
    been left out of the diagram. 
    
    \indentpar Examining the \textbf{ModelOutputBehaviour} interface is sufficient
    to understand the implementation of the excluded interfaces, while the \textbf{CalculationOutputBehaviour}
    is more complex than the other interfaces. The classes that implement \textbf{CalculationOutputBehaviour}
    are responsible for making method specific calculations, and return a set of values that is dependant on 
    the method itself. This is why the \textbf{CalculationOutputBehaviour} \emph{getValue()} function is said to
    return a \textbf{Tuple}. This \textbf{Tuple} contains different info based on the specified model, and classes
    that access this \textbf{Tuple} must be prepared to parse it.

    \indentpar The design pattern of the \emph{Output} class and the interfaces it uses (all the interfaces that 
    end in "\emph{Behaviour}") is inspired by the \emph{Duck example} in the opening chapter of \href{https://github.com/ksatria/MK-Design-Pattern/blob/master/Ebook/Head%20First%20Design%20Patterns.pdf}{this textbook}
    on design patterns.
    %%%%%%%%%%%%%%%%%%%%%%%%%%%%%%%%%%%%%%%%%%%%%%%%%%%%%%%%%%%%%%%

\end{document}