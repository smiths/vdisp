\documentclass[12pt, titlepage]{article}

\usepackage{amsmath, mathtools}

\usepackage[round]{natbib}
\usepackage{amsfonts}
\usepackage{amssymb}
\usepackage{graphicx}
\usepackage{colortbl}
\usepackage{xr}
\usepackage{hyperref}
\usepackage{longtable}
\usepackage{xfrac}
\usepackage{tabularx}
\usepackage{float}
\usepackage{siunitx}
\usepackage{booktabs}
\usepackage{multirow}
\usepackage[section]{placeins}
\usepackage{caption}
\usepackage{fullpage}

\hypersetup{
colorlinks=true,       % false: boxed links; true: colored links
linkcolor=red,          % color of internal links (change box color with linkbordercolor)
citecolor=blue,      % color of links to bibliography
filecolor=magenta,  % color of file links
urlcolor=cyan          % color of external links
}

\usepackage{array}

\begin{document}

\title{Module Interface Specification for VDisp}
\author{Emil Soleymani, Dr.~Spencer Smith}
\date{\today}
\maketitle

\section{Revision History}

\begin{tabularx}{\textwidth}{p{3cm}p{2cm}X}
\toprule {\bf Date} & {\bf Version} & {\bf Notes}\\
\midrule
May 18, 2022 & 1.0 & Initial Draft\\
\bottomrule
\end{tabularx}

~\newpage

\section{Symbols, Abbreviations, and Acronyms}
\emph{\# TODO}
\newpage

\tableofcontents

\newpage

\pagenumbering{arabic}

\section{Introduction}

The following document details the Module Interface Specifications for \emph{VDisp}

Complementary documents include the System Requirement Specifications
and Module Guide.  The full documentation and implementation can be
found at \url{...}.  

\section{Notation}


The structure of the MIS for modules comes from ,
with the addition that template modules have been adapted from
.  The mathematical notation comes from Chapter 3 of
.  For instance, the symbol := is used for a
multiple assignment statement and conditional rules follow the form $(c_1
\Rightarrow r_1 | c_2 \Rightarrow r_2 | ... | c_n \Rightarrow r_n )$.

The following table summarizes the primitive data types used by VDisp. 

\begin{center}
\renewcommand{\arraystretch}{1.2}
\noindent 
\begin{tabular}{l l p{7.5cm}} 
\toprule 
\textbf{Data Type} & \textbf{Notation} & \textbf{Description}\\ 
\midrule
character & char & a single symbol or digit\\
integer & $\mathbb{Z}$ & a number without a fractional component in (-$\infty$, $\infty$) \\
natural number & $\mathbb{N}$ & a number without a fractional component in [1, $\infty$) \\
real & $\mathbb{R}$ & any number in (-$\infty$, $\infty$)\\
\bottomrule
\end{tabular} 
\end{center}

\noindent
The specification of VDisp uses some derived data types: sequences, strings, and
tuples. Sequences are lists filled with elements of the same data type. Strings
are sequences of characters. Tuples contain a list of values, potentially of
different types. In addition, VDisp uses functions, which
are defined by the data types of their inputs and outputs. Local functions are
described by giving their type signature followed by their specification.

\section{Module Decomposition}

The following table is taken directly from the Module Guide document for this project.

\begin{table}[h!]
\centering
\begin{tabular}{p{0.3\textwidth} p{0.6\textwidth}}
\toprule
\textbf{Level 1} & \textbf{Level 2}\\
\midrule

{Hardware-Hiding Module} & ~ \\
\midrule

\multirow{5}{0.3\textwidth}{Behaviour-Hiding Module} & Input Parameters Module\\
& Output Format Module\\
& Equation Module\\
& Control Module\\
& Display Module \\
\midrule

{Software-Decision Module} & Plotting Module\\
\bottomrule

\end{tabular}
\caption{Module Hierarchy}
\label{TblMH}
\end{table}
  
\newpage
~\newpage

\section{MIS of Input Parameters Module} \label{InputParametersModule} 

\subsection{Module}
InputFormat
\subsection{Uses}
None
\subsection{Syntax}

\subsubsection{Exported Types}
InputData\\
Model \\
Foundation
\subsubsection{Exported Access Programs}

\emph{\# TODO: Add exceptions}
\begin{center}
\begin{tabular}{p{2cm} p{4cm} p{4cm} p{2cm}}
\hline
\textbf{Name} & \textbf{In} & \textbf{Out} & \textbf{Exceptions} \\
\hline
InputData & String & InputData & none \\
\hline
\end{tabular}
\end{center}

\subsection{Semantics}

\subsubsection{State Variables}

\emph{TODO: Copy these over from SRS once made}

\subsubsection{Environment Variables}

inputFile: String[] \emph{\#f[i] is the ith string in text file f}

\subsubsection{Assumptions}

\emph{\# TODO: comments in input file are denoted with a "\#"}
\begin{itemize}
  \item Input file is in correct file format
  \item Input file will be fully parsed, with its corresponding values 
  stored properly in an InputData instance before any attempts to access 
  them
\end{itemize}

\subsubsection{Access Routine Semantics}

\noindent InputData(s):
\begin{itemize}
\item transition: 
The filename $s$ is first associated with input file $f$, which is
used to modify state variables using the following procedural specification:
  \begin{enumerate}
    \item Read data sequentially from $f$ to populate the state variables
    from \emph{\# TODO: insert requirement here}
    \item Calculate the derived quantities (all other state variables) as follows:
      \begin{itemize}
        \item \emph{\# TODO: list derived values}
        \item \emph{e.x. model:Model is derived from modelOutput:Int}
      \end{itemize}
    \item \emph{\# TODO: Come up with constraints on variables (e.x $nodalPoints > 0$, 
    $0.5 \le poissoinsRatio \le 1$) then make local function verifyParams()}
  \end{enumerate}
\item output: $out := $ inputData:InputData where inputData has parameters specified from file $f$ 
\item exception: $exc := $ a file named $s$ cannot be found OR the format of file $f$ is incorrect 
$\Rightarrow$ FileError
\end{itemize}

\subsubsection{Local Functions}

$\text{validFile}(S): String \rightarrow \mathbb{B}$
\begin{itemize}
  \item output: $out := true$ iff file $S$ is in valid format
  \item exception: $exc := $ a file named $s$ cannot be found OR the format of file $f$ is incorrect 
  $\Rightarrow$ FileError
\end{itemize}

\noindent $\text{verifyParams}(inputData): \text{InputData} \rightarrow \mathbb{B}$
\begin{itemize}
  \item output: $out := $ none
  \item exception: $exc := $ 
  \begin{align*}
  \neg(inputData.nodalPoints > 0) &\Rightarrow \text{badNodalPoints} \\
  \neg(0 \le inputData.modelOption \le 4) &\Rightarrow \text{badModelOption} \\
  \emph{etc...} \\
  \end{align*}
\end{itemize}

\subsection{Considerations}
The value of each state variable can be accessed through its name 
(getter). An access program is available for each state variable. 
There are no setters for the state variables, since the values will 
be set \emph{(and checked)} by the InputData constructor and not changed
for the life of the InputData instance. 

\pagebreak
\section{MIS of Control Module} \label{ControlModule} 

\subsection{Module}
main
\subsection{Uses}
InputFormat, OutputFormat
\subsection{Syntax}

\subsubsection{Exported Constants}

\subsubsection{Exported Access Programs}

\begin{center}
\begin{tabular}{p{2cm} p{4cm} p{4cm} p{2cm}}
\hline
\textbf{Name} & \textbf{In} & \textbf{Out} & \textbf{Exceptions} \\
\hline
main & - & - & - \\
\hline
\end{tabular}
\end{center}

\subsection{Semantics}

\subsubsection{State Variables}
None.

\subsubsection{Access Routine Semantics}

\noindent main():
\begin{itemize}
\item transition: Modify state of InputFormat module and the 
environment variables for the \emph{Plot?} and OutputFormat 
modules by following these steps:
  \begin{itemize}
    \item Get (inputFile:String) and (outputFile:string) from user
    \item \emph{\# TODO: List steps}
  \end{itemize}
\end{itemize}

\pagebreak
\section{MIS of Output Format Module} \label{OutputFormatModule} 

\subsection{Module}
OutputFormat
\subsection{Uses}
InputFormat, Equation \emph{\# TODO: Not specified yet}
\subsection{Syntax}

\subsubsection{Exported Types}
OutputData
\subsubsection{Exported Access Programs}
% \begin{tabular}{p{4cm} p{5cm} p{2cm} p{3.5cm}}
\begin{center}
\begin{tabular}{p{0.25\textwidth} p{0.25\textwidth} p{0.15\textwidth} p{0.2\textwidth}}
\hline
\textbf{Name} & \textbf{In} & \textbf{Out} & \textbf{Exceptions} \\
\hline
OutputData & String & OutputData & FileError \\
writeOutput & OutputData,String, Function[] & None & TODO \\
writeHeader & OutputData,String & None & TODO \\
.... & .... & .... & .... \\
performWriteModel Output & OutputData,String & None & TODO \\
.... & .... & .... & .... \\
\hline
\end{tabular}
\end{center}

\subsection{Semantics}

\subsubsection{State Variables}
inputData:InputData

\subsubsection{Environment Variables}
outputFile:string
\subsubsection{Access Routine Semantics}

\noindent OutputData(s):
\begin{itemize}
\item transition: this.inputData := InputData(s)
\item output: $out :=  $ this
\end{itemize}

\noindent writeOutput(outputData, s, funcs):
\begin{itemize}
\item transition: The filename $s$ is first associated with output file $out$, which is
used to write state variables using the following procedural specification:
  \begin{enumerate}
    \item Clear any existing contents of file $out$
    \item For each function $f$ in funcs
      \begin{itemize}
        \item Call function $f$, passing in outputData as argument
        \item Write return value of $f(\text{outputData})$ to file $out$
      \end{itemize}
  \end{enumerate}
\end{itemize}

\noindent writeHeader(outputData, s):
\begin{itemize}
  \item transition: The filename $s$ is first associated with output file $out$, which is
  used to write state variables using the following procedural specification:
    \begin{enumerate}
      \item Append “Title: “$\|$outputData.inputData.problemName to file $out$
      \item Append “Nodal Points: “$\|$outputData.inputData.nodalPoints to file $out$
      \item Append “Base Nodal Point Index: “$\|$outputData.inputData.bottomPointIndex to file $out$
      \item Append “Number of Different Soil Layers: “$\|$outputData.inputData.soilLayers to file $out$
      \item Append “Increment Depth: “$\|$outputData.inputData.dx to file $out$
    \end{enumerate}
  \end{itemize}

\medskip
\emph{\# TODO: Specify rest of the functions}

\subsubsection{Local Functions}
None.

\subsection{Considerations}
$x \| y$ denotes the concatenation of strings $x$ and $y$

\pagebreak
\section{MIS of Module Name} \label{Module} 

\subsection{Module}

\subsection{Uses}

\subsection{Syntax}

\subsubsection{Exported Constants}

\subsubsection{Exported Access Programs}

\begin{center}
\begin{tabular}{p{2cm} p{4cm} p{4cm} p{2cm}}
\hline
\textbf{Name} & \textbf{In} & \textbf{Out} & \textbf{Exceptions} \\
\hline
accessProg & - & - & - \\
\hline
\end{tabular}
\end{center}

\subsection{Semantics}

\subsubsection{State Variables}

Not all modules will have state variables.  State variables give the module
  a memory.

\subsubsection{Environment Variables}

This section is not necessary for all modules.  Its purpose is to capture
  when the module has external interaction with the environment, such as for a
  device driver, screen interface, keyboard, file, etc.

\subsubsection{Assumptions}

Try to minimize assumptions and anticipate programmer errors via
  exceptions, but for practical purposes assumptions are sometimes appropriate.

\subsubsection{Access Routine Semantics}

\noindent accessProg():
\begin{itemize}
\item transition: if appropriate
\item output: 
\item exception:  
\end{itemize}

A module without environment variables or state variables is unlikely to
  have a state transition.  In this case a state transition can only occur if
  the module is changing the state of another module.

Modules rarely have both a transition and an output.  In most cases you
  will have one or the other.

\subsubsection{Local Functions}

As appropriate: These functions are for the purpose of specification.
  They are not necessarily something that is going to be implemented
  explicitly.  Even if they are implemented, they are not exported; they only
  have local scope.

\newpage

\bibliographystyle {plainnat}
\bibliography {../../../refs/References}

\newpage

\section{Appendix} \label{Appendix}

\begin{table}[h!]
  \caption{Possible Exceptions}
  \label{TblExceptions}
  \centering
  \begin{tabular}{p{0.3\textwidth} p{0.6\textwidth}}
  \toprule
  \textbf{Message ID} & \textbf{Error Message}\\
  \midrule
  {FileError} & Error: File not found \\
  {badNodalPoints} & Error: Number of nodal points must be $>$ 0 \\
  ... & ... \\
  {warnDX} & Error: It is recommended that $0.01 \le dx \le 3$\\
  ... & ... \\
  TODO & TODO \\
  \bottomrule
  \end{tabular}
  \end{table}
\end{document}