\documentclass[12pt]{article}

\usepackage{amsmath, mathtools}
\usepackage{amsfonts}
\usepackage{amssymb}
\usepackage{graphicx}
\usepackage{colortbl}
\usepackage{xr}
\usepackage{hyperref}
\usepackage{longtable}
\usepackage{xfrac}
\usepackage{tabularx}
\usepackage{float}
\usepackage{siunitx}
\usepackage{booktabs}
\usepackage{caption}
\usepackage{pdflscape}
\usepackage{afterpage}
\usepackage{enumitem}

\usepackage[round]{natbib}

\usepackage{fullpage}

\newcommand{\indentpar}{\phantom{=}}

%\usepackage{refcheck}
\usepackage{xcolor}
\hypersetup{
    colorlinks,
    linkcolor={red!50!black},
    citecolor={blue!50!black},
    urlcolor={blue!80!black}
}

% For easy change of table widths
\newcommand{\colZwidth}{1.0\textwidth}
\newcommand{\colAwidth}{0.13\textwidth}
\newcommand{\colBwidth}{0.82\textwidth}
\newcommand{\colCwidth}{0.1\textwidth}
\newcommand{\colDwidth}{0.05\textwidth}
\newcommand{\colEwidth}{0.8\textwidth}
\newcommand{\colFwidth}{0.17\textwidth}
\newcommand{\colGwidth}{0.5\textwidth}
\newcommand{\colHwidth}{0.28\textwidth}

% Used so that cross-references have a meaningful prefix
% TODO - remove these in favour of hyperrefs with labels and phantomsections
\newcounter{defnum} %Definition Number
\newcommand{\dthedefnum}{GD\thedefnum}
\newcommand{\dref}[1]{GD\ref{#1}}
\newcounter{datadefnum} %Datadefinition Number
\newcommand{\ddthedatadefnum}{DD\thedatadefnum}
\newcommand{\ddref}[1]{DD\ref{#1}}
\newcounter{theorynum} %Theory Number
\newcommand{\tthetheorynum}{T\thetheorynum}
\newcommand{\tref}[1]{T\ref{#1}}
\newcounter{tablenum} %Table Number
\newcommand{\tbthetablenum}{T\thetablenum}
\newcommand{\tbref}[1]{TB\ref{#1}}
\newcounter{assumpnum} %Assumption Number
\newcommand{\atheassumpnum}{P\theassumpnum}
\newcommand{\aref}[1]{A\ref{#1}}
\newcounter{goalnum} %Goal Number
\newcommand{\gthegoalnum}{P\thegoalnum}
\newcommand{\gsref}[1]{GS\ref{#1}}
\newcounter{instnum} %Instance Number
\newcommand{\itheinstnum}{IM\theinstnum}
\newcommand{\iref}[1]{IM\ref{#1}}
\newcounter{reqnum} %Requirement Number
\newcommand{\rthereqnum}{P\thereqnum}
\newcommand{\rref}[1]{R\ref{#1}}
\newcounter{nfrnum} %NFR Number
\newcommand{\rthenfrnum}{NFR\thenfrnum}
\newcommand{\nfrref}[1]{NFR\ref{#1}}
\newcounter{lcnum} %Likely change number
\newcommand{\lthelcnum}{LC\thelcnum}
\newcommand{\lcref}[1]{LC\ref{#1}}

\newlist{symbDescription}{description}{1}
\setlist[symbDescription]{noitemsep, topsep=0pt, parsep=0pt, partopsep=0pt}

\newcommand{\deftheory}[9][Not Applicable]
{
\newpage
\noindent \rule{\textwidth}{0.5mm}

\paragraph{RefName: } \textbf{#2} \phantomsection 
\label{#2}

\paragraph{Label:} #3

\noindent \rule{\textwidth}{0.5mm}

\paragraph{Equation:}

#4

\paragraph{Description:}

#5

\paragraph{Notes:}

#6

\paragraph{Source:}

#7

\paragraph{Ref.\ By:}

#8

\paragraph{Preconditions for \hyperref[#2]{#2}:}
\label{#2_precond}

#9

\paragraph{Derivation for \hyperref[#2]{#2}:}
\label{#2_deriv}

#1

\noindent \rule{\textwidth}{0.5mm}

}

%%%%%%%%%%%%%%%%%%%%%%%%%%%%%%%%%%%%%%

\begin{document}

\title{Software Requirements Specification for VDisp} 
\author{Emil Soleymani, Dr.\ Spencer Smith}
\date{\today}
	
\maketitle

~\newpage

\pagenumbering{roman}

\tableofcontents

~\newpage

\section*{Revision History}

\begin{tabularx}{\textwidth}{p{3cm}p{2cm}X}
\toprule {\bf Date} & {\bf Version} & {\bf Notes}\\
\midrule
May 24, 2022 & 1.0 & Initial Draft\\
\bottomrule
\end{tabularx}

~\newpage

\section{Reference Material}

This section records information for easy reference.

\subsection{Table of Units}

Throughout this document SI (Syst\`{e}me International d'Unit\'{e}s) is employed
as the unit system.  In addition to the basic units, several derived units are
used as described below.  For each unit, the symbol is given followed by a
description of the unit and the SI name.

\renewcommand{\arraystretch}{1.2}
%\begin{table}[ht]
  \noindent \begin{tabular}{l l l} 
    \toprule		
    \textbf{Symbol} & \textbf{Unit} & \textbf{SI}\\
    \midrule 
    \si{\metre} & length & meter\\
    \si{\kilogram} & mass	& kilogram\\
    \si{\second} & time & second\\
    \si{\radian} & angle & radians\\
    \si{\newton}=\si{\kilogram\meter\per\square\second} & force & newton\\
    \si{\pascal}=\si{\newton\per\square\meter} & pressure & pascal\\
    \emph{Add imperial} units & ... & ..\\
    \bottomrule
  \end{tabular}
  %	\caption{Provide a caption}
%\end{table}

\subsection{Table of Symbols}

\noindent \begin{tabular}{l p{12cm} l} 
  \toprule		
  \textbf{Symbol} & \textbf{Description} & \textbf{Units}\\
  \midrule 
  $e$ & Void ratio & ? \\
  $K_0$ & Coefficient of lateral earth pressure & ?\\
  $\sigma_v$ & Vertical stress & meter\\
  $\sigma_{v}'$ & Effective stress & meter\\
  $\gamma_w$ & Unit weight of water & \si{\newton\per\cubic\metre} \\
  $\phi_s$ & Plane strain angle & \si{\radian}\\
  \bottomrule
\end{tabular}
\\
\emph{TODO: Add all constants, arrange in alphabetical order, add caption for table?}
\subsection{Abbreviations and Acronyms}

\renewcommand{\arraystretch}{1.2}
\begin{tabular}{l l} 
  \toprule		
  \textbf{symbol} & \textbf{description}\\
  \midrule 
  A & Assumption\\
  DD & Data Definition\\
  GD & General Definition\\
  GS & Goal Statement\\
  IM & Instance Model\\
  LC & Likely Change\\
  PS & Physical System Description\\
  R & Requirement\\
  SRS & Software Requirements Specification\\
  \emph{VDisp} & Software Settlement Analysis Software\\
  T & Theoretical Model\\
  \bottomrule
\end{tabular}\\

\emph{Should we add MG to this? Or MIS?}

\subsection{Mathematical Notation}

This section is optional, but should be included for projects that make use of
notation to convey mathematical information.  For instance, if typographic
conventions (like bold face font) are used to distinguish matrices, this should
be stated here.  If symbols are used to show mathematical operations, these
should be summarized here.  In some cases the easiest way to summarize the
notation is to point to a text or other source that explains the notation.

This section was added to the template because some students use very domain
specific notation.  This notation will not be readily understandable to people
outside of your domain.  It should be explained.

\newpage

\pagenumbering{arabic}

This SRS template is based on \citet{SmithAndLai2005, SmithEtAl2007}.  It will
get you started.  You should not modify the section headings, without first
discussing the change with the course instructor.  Modification means you are
not following the template, which loses some of the advantage of a template,
especially standardization.  Although the bits shown below do not include type
information, you may need to add this information for your problem.  If you are
unsure, please can ask the instructor.

Feel free to change the appearance of the report by modifying the LaTeX
commands.

This template document assumes that a single program is being documented. If you
are documenting a family of models, you should start with a commonality
analysis.  A separate template is provided for this.  For program families you
should look at \cite{Smith2006, SmithMcCutchanAndCarette2017}. Single family
member programs are often programs based on a single physical model.  General
purpose tools are usually documented as a family.  Families of physical models
also come up.

The SRS is not generally written, or read, sequentially.  The SRS is a reference
document.  It is generally read in an ad hoc order, as the need arises.  For
writing an SRS, and for reading one for the first time, the suggested order of
sections is:

\begin{itemize}
\item Goal Statement
\item Instance Models
\item Requirements
\item Introduction
\item Specific System Description
\end{itemize}

Guiding principles for the SRS document:
\begin{itemize}
\item Do not repeat the same information at the same abstraction level.  If
  information is repeated, the repetition should be at a different abstraction
  level.  For instance, there will be overlap between the scope section and the
  assumptions, but the scope section will not go into as much detail as the
  assumptions section.
\end{itemize}


The template description comments should be disabled before submitting this
document for grading.

You can borrow any wording from the text given in the template.  It is part of
the template, and not considered an instance of academic integrity.  Of course,
you need to cite the source of the template.

When the documentation is done, it should be possible to trace back to the
source of every piece of information.  Some information will come from external
sources, like terminology.  Other information will be derived, like General
Definitions.

An SRS document should have the following qualities: unambiguous, consistent,
complete, validatable, abstract and traceable.

The overall goal of the SRS is that someone that meets the Characteristics of
the Intended Reader (Section~\ref{sec_IntendedReader}) can learn, understand and
verify the captured domain knowledge.  They should not have to trust the authors
of the SRS on any statements.  They should be able to independently
verify/derive every statement made.

\section{Introduction}

\indentpar \indentpar Our cities and towns are in constant need of new
infrastructure due to the expanding and evolving nature of our societies. With
architects planning new buildings every day, there is a never ending need for
experienced geotechnical engineers to perform the complex task of soil
settlement analysis prior to laying down the foundation. The \emph{VDisp}
software aims to provide undergraduate Civil Engineering students the tools they
need to make the simple analyses expected from them in their studies, while also
providing them with visualizations and real-time interaction as a means of
educational enhancement.

The following section provides an overview of the Software Requirements
Specification (SRS) for the soil settlement analysis software. The developed
program will be referred to as \emph{VDisp}. This section explains the purpose
of this document, the scope of the requirements, the characteristics of the
intended reader, and the organization of the document.


\subsection{Purpose of Document}

\indentpar \indentpar The primary purpose of this document is to record the requirements of 
the \emph{VDisp} software. Goals, assumptions, theoretical models, 
definitions, and other model derivation information are specified, allowing 
the reader to fully understand and verify the purpose and scientific basis 
of \emph{VDisp}. Except for system constraints, this SRS will 
remain abstract, describing what problem the software is solving, but not how to 
solve it.

This document will be used as a starting point for subsequent development
phases, including writing the design specification and the software verification
and validation plan. The design document will show how the requirements are to
be realized, including decisions on the numerical algorithms and programming
environment. The verification and validation plan will show the steps that will
be used to increase confidence in the software documentation and the
implementation. Although the SRS fits in a series of documents that follow the
so-called waterfall model, the actual development process is not constrained in
any way. Even when the waterfall model is not followed, as Parnas and Clements
point out parnasClements1986(\emph{TODO: add citation}), the most logical way to
present the documentation is still to ``fake'' a rational design process.

\subsection{Scope of Requirements} 

The scope of the requirements includes soil settlement analysis of rectangular
slab foundations and long strip footing. This entire document is written
assuming homogeneous\emph{(is this how you say same material throughout layer?)}
soil layers and is restricted to one-dimensional analysis \emph{(for now)}.

\subsection{Characteristics of Intended Reader} \label{sec_IntendedReader}

\emph{TODO}

This section summarizes the skills and knowledge of the readers of the SRS.  It
does NOT have the same purpose as the ``User Characteristics'' section
(Section~\ref{SecUserCharacteristics}).  The intended readers are the people
that will read, review and maintain the SRS.  They are the people that will
conceivably design the software that is intended to meet the requirements.  The
user, on the other hand, is the person that uses the software that is built.
They may never read this SRS document.  Of course, the same person could be a
``user'' and an ``intended reader.''

The intended reader characteristics should be written as unambiguously and as
specifically as possible.  Rather than say, the user should have an
understanding of physics, say what kind of physics and at what level.  For
instance, is high school physics adequate, or should the reader have had a
graduate course on advanced quantum mechanics?

\subsection{Organization of Document}

\emph{TODO}

This section provides a roadmap of the SRS document.  It will help the reader
orient themselves.  It will provide direction that will help them select which
sections they want to read, and in what order.  This section will be similar
between project.

\section{General System Description}

\emph{TODO}

This section provides general information about the system.  It identifies the
interfaces between the system and its environment, describes the user
characteristics and lists the system constraints.  This text can likely be
borrowed verbatim.

The purpose of this section is to provide general information about the system
so the specific requirements in the next section will be easier to understand.
The general system description section is designed to be changeable independent
of changes to the functional requirements documented in the specific system
description. The general system description provides a context for a family of
related models. The general description can stay the same, while specific
details are changed between family members.

\subsection{System Context}

\emph{TODO}

Your system context will include a figure that shows the abstract view of the
software.  Often in a scientific context, the program can be viewed abstractly
following the design pattern of Inputs $\rightarrow$ Calculations $\rightarrow$
Outputs.  The system context will therefore often follow this pattern.  The user
provides inputs, the system does the calculations, and then provides the outputs
to the user.  The figure should not show all of the inputs, just an abstract
view of the main categories of inputs (like material properties, geometry,
etc.).  Likewise, the outputs should be presented from an abstract point of
view.  In some cases the diagram will show other external entities, besides the
user.  For instance, when the software product is a library, the user will be
another software program, not an actual end user. If there are system
constraints that the software must work with external libraries, these libraries
can also be shown on the System Context diagram. They should only be named with
a specific library name if this is required by the system constraint.

\begin{figure}[h!]
\begin{center}
  \emph{TODO}
\caption{System Context}
\label{Fig_SystemContext} 
\end{center}
\end{figure}

For each of the entities in the system context diagram its responsibilities
should be listed.  Whenever possible the system should check for data quality,
but for some cases the user will need to assume that responsibility.  The list
of responsibilites should be about the inputs and outputs only, and they should
be abstract.  Details should not be presented here.  However, the information
should not be so abstract as to just say ``inputs'' and ``outputs''.  A
summarizing phrase can be used to characterize the inputs. For instance, saying
``material properties'' provides some information, but it stays away from the
detail of listing every required properties.

\begin{itemize}
\item User Responsibilities:
\begin{itemize}
\item 
\end{itemize}
\item \emph{VDisp}{} Responsibilities:
\begin{itemize}
\item Detect data type mismatch, such as a string of characters instead of a
  floating point number
\item 
\end{itemize}
\end{itemize}

\subsection{User Characteristics} \label{SecUserCharacteristics}

\emph{TODO}

This section summarizes the knowledge/skills expected of the user. Measuring
usability, which is often a required non-function requirement, requires
knowledge of a typical user.  As mentioned above, the user is a different role
from the ``intended reader,'' as given in Section~\ref{sec_IntendedReader}.  As
in Section~\ref{sec_IntendedReader}, the user characteristics should be specific
an unambiguous. For instance, ``The end user of \emph{VDisp}{} should have an
understanding of undergraduate Level 1 Calculus and Physics.''

\subsection{System Constraints}

\emph{TODO}

System constraints differ from other type of requirements because they limit the
developers’ options in the system design and they identify how the eventual
system must fit into the world. This is the only place in the SRS where design
decisions can be specified.  That is, the quality requirement for abstraction is
relaxed here.  However, system constraints should only be included if they are
truly required. In the context of CAS 741, you often will may not have any
system constraints.

\section{Specific System Description}

\emph{TODO}

This section first presents the problem description, which gives a high-level
view of the problem to be solved.  This is followed by the solution characteristics
specification, which presents the assumptions, theories, definitions and finally
the instance models.  Add any project specific details that are relevant
for the section overview.

\subsection{Problem Description} \label{Sec_pd}

\emph{TODO}

\emph{VDisp} is intended to solve ... What problem does your program solve?
The description here should be in the problem space, not the solution space.

\subsubsection{Terminology and  Definitions}

\emph{TODO}

This section is expressed in words, not with equations.  It provide the meaning
of the different words and phrases used in the domain of the problem. The
terminology is used to introduce concepts from the world outside of the
mathematical model  The terminology provides a real world connection to give the
mathematical model meaning.

This subsection provides a list of terms that are used in the subsequent
sections and their meaning, with the purpose of reducing ambiguity and making it
easier to correctly understand the requirements:

\begin{itemize}

\item 

\end{itemize}

\subsubsection{Physical System Description} \label{sec_phySystDescrip}

\emph{TODO}

The purpose of this section is to clearly and unambiguously state the physical
system that is to be modelled. Effective problem solving requires a logical and
organized approach. The statements on the physical system to be studied should
cover enough information to solve the problem. The physical description involves
element identification, where elements are defined as independent and separable
items of the physical system. Some example elements include acceleration due to
gravity, the mass of an object, and the size and shape of an object. Each
element should be identified and labelled, with their interesting properties
specified clearly. The physical description can also include interactions of the
elements, such as the following: i) the interactions between the elements and
their physical environment; ii) the interactions between elements; and, iii) the
initial or boundary conditions.

The physical system of \emph{VDisp}{}, as shown in Figure~?,
includes the following elements:

\begin{itemize}

\item[PS1:] 

\item[PS2:] ...

\end{itemize}

A figure here makes sense for most SRS documents

% \begin{figure}[h!]
% \begin{center}
% %\rotatebox{-90}
% {
%  \includegraphics[width=0.5\textwidth]{<FigureName>}
% }
% \caption{\label{<Label>} <Caption>}
% \end{center}
% \end{figure}

\subsubsection{Goal Statements}

\emph{TODO}

The goal statements refine the ``Problem Description'' (Section~\ref{Sec_pd}).
A goal is a functional objective the system under consideration should achieve.
Goals provide criteria for sufficient completeness of a requirements
specification and for requirements pertinence. Goals will be refined in Section
“Instanced Models” (Section~\ref{sec_instance}). Large and complex goals should
be decomposed into smaller sub-goals.  The goals are written abstractly, with a
minimal amount of technical language.  They should be understandable by
non-domain experts.

\noindent Given the inputs, the goal statements are:

\begin{itemize}

\item[GS:FindDisp:\phantomsection\label{GS:FindDisp}]: Given the soil
properties, load, and type of footing, find the vertical displacement of the
footing.

\end{itemize}

\subsection{Solution Characteristics Specification}

\emph{TODO}

This section specifies the information in the solution domain of the system to
be developed. This section is intended to express what is required in such a way
that analysts and stakeholders get a clear picture, and the latter will accept
it. The purpose of this section is to reduce the problem into one expressed in
mathematical terms. Mathematical expertise is used to extract the essentials
from the underlying physical description of the problem, and to collect and
substantiate all physical data pertinent to the problem.

This section presents the solution characteristics by successively refining
models.  It starts with the abstract/general Theoretical Models (TMs) and
refines them to the concrete/specific Instance Models (IMs).  If necessary there
are intermediate refinements to General Definitions (GDs).  All of these
refinements can potentially use Assumptions (A) and Data Definitions (DD). TMs
are refined to create new models, that are called GMs or IMs. DDs are not
refined; they are just used. GDs and IMs are derived, or refined, from other
models. DDs are not derived; they are just given. TMs are also just given, but
they are refined, not used.  If a potential DD includes a derivation, then that
means it is refining other models, which would make it a GD or an IM.

The above makes a distinction between ``refined'' and ``used.'' A model is
refined to another model if it is changed by the refinement. When we change a
general 3D equation to a 2D equation, we are making a refinement, by applying
the assumption that the third dimension does not matter. If we use a definition,
like the definition of density, we aren't refining, or changing that definition,
we are just using it.

The same information can be a TM in one problem and a DD in another.  It is
about how the information is used.  In one problem the definition of
acceleration can be a TM, in another it would be a DD.

There is repetition between the information given in the different chunks (TM,
GDs etc) with other information in the document.  For instance, the meaning of
the symbols, the units etc are repeated.  This is so that the chunks can stand
on their own when being read by a reviewer/user.  It also facilitates reuse of
the models in a different context.

The relationships between the parts of the document are show in the
following figure.  In this diagram ``may ref'' has the same role as ``uses''
above.  The figure adds ``Likely Changes,'' which are able to reference (use)
Assumptions.

\begin{figure}[H]
\end{figure}

The instance models that govern \emph{VDisp}{} are presented in
Subsection~\ref{sec_instance}.  The information to understand the meaning of the
instance models and their derivation is also presented, so that the instance
models can be verified.

\subsubsection{Assumptions} \label{sec_assumpt}

\emph{TODO}

The assumptions are a refinement of the scope.  The scope is general, where the
assumptions are specific.  All assumptions should be listed, even those that
domain experts know so well that they are rarely (if ever) written down. The
document should not take for granted that the reader knows which assumptions
have been made. In the case of unusual assumptions, it is recommended that the
documentation either include, or point to, an explanation and justification for
the assumption.

This section simplifies the original problem and helps in developing the
theoretical model by filling in the missing information for the physical
system. The numbers given in the square brackets refer to the theoretical model
[T], general definition [GD], data definition [DD], instance model [IM], or
likely change [LC], in which the respective assumption is used.

\begin{itemize}
  
  \item[A:SLH:\phantomsection\label{A:SLH}]{The soil mass is homogeneous,
  with consistent soil properties throughout. (RefBy:
  \hyperref[LC_inhomogeneous]{LC:Calculate-Inhomogeneous-Soil-Layers}.)}

  \item[A:Isotropic:\phantomsection\label{A:Isotropic}]{The material properties are independent of direction. (RefBy: ?.)}

  \item [A:Saturated:\phantomsection\label{A:Saturated}]{The soil properties are
  independent of dry or saturated conditions, with the exception of unit weight.
  (RefBy: ?.)}
          
  \item [A:HalfPlane:\phantomsection\label{A:HalfPlane}]{The domain is half of a
  2D Euclidean plane, as divided by a straight line. (RefBy: ?.)}

  \item [A:Terzaghi:\phantomsection\label{A:Terzaghi}]{When we apply stress to a
  porous material, the stress is opposed by the pressure in the fluid that fills
  the pores of the material \citep{WikipediaTerzaghi2022}. (RefBy: ?.)}

  \item [A:Continuum:\phantomsection\label{A:Continuum}]{The underlying molecular structure of matter is not considered and gaps and empty spaces within a material particle are ignored. The material is assumed to be continuous. \citep[p.\ 33--34]{Long1961}, \citep[p.\ 1-2]{Malvern1969} (RefBy: ?.)}

\end{itemize}
  
\subsubsection{Theoretical Models}\label{sec_theoretical}

\emph{TODO}

Theoretical models are sets of abstract mathematical equations or axioms for
solving the problem described in Section ``Physical System Description''
(Section~\ref{sec_phySystDescrip}). Examples of theoretical models are physical
laws, constitutive equations, relevant conversion factors, etc.

This section focuses on the general equations and laws that \emph{VDisp}{} is
based on.  Modify the examples below for your problem, and add additional models
as appropriate.

\bigskip

%%%%%%%%%%%%%%%%%%%%%%%%%%%%%%%%%%%%%%

\deftheory
% #2 refname of theory
{BT:NormStress}
% #3 label
{Total Normal Stress}
% #4 equation
{
\( \sigma = \frac{F_n}{A}\)
}
% #5 description
{
\begin{symbDescription}
  \item{$\sigma$ is the total normal stress (\si{\pascal})}
  \item{${F_{\text{n}}}$ is the total normal force (\si{\newton})}
  \item{$A$ is the cross-sectional area (\si{\metre\squared})}
\end{symbDescription}
}
% #6 Notes
{
None.
}
% #7 Source
{
\citet{FredlundKrahn}
}
% #8 Referenced by
{
\hyperref[BT:EffStress]{BT:EffStress}
}
% #9 Preconditions
{
\begin{itemize}
\item list preconditions here
\end{itemize}
}
% #1 derivation - not applicable by default
{}

%%%%%%%%%%%%%%%%%%%%%%%%%%%%%%%%%%%%%%

\deftheory
% #2 refname of theory
{BT:EffStress}
% #3 label
{Effective Stress}
% #4 equation
{
\( \sigma' =\sigma - u \)
}
% #5 description
{
\begin{symbDescription}
\item $\sigma$ is the total normal stress on the soil mass (\si{\pascal}).
\item $\sigma'$ is the effective normal stress provided by the soil skeleton
(\si{\pascal}). 
\item $u$ is the pore pressure from the water within the soil (\si{\pascal}).
\end{symbDescription}
}
% #6 Notes
{ According to Terzaghi's principle (\hyperref[A:Terzaghi]{A:Terzaghi}) the
changes caused by stress in a porous medium are a result of changes to the
effective stress. The total stress $\sigma$ is defined in
\hyperref[BT:NormStress]{BT:NormStress}. }
% #7 Source
{
\citet{FredlundKrahn}
}
% #8 Referenced by
{
\hyperref[label]{text}
}
% #9 Preconditions
{
\begin{itemize}
\item \hyperref[A:Terzaghi]{A:Terzaghi}
\end{itemize}
}
% #1 derivation - not applicable by default
{}

%%%%%%%%%%%%%%%%%%%%%%%%%%%%%%%%%%%%%%

\deftheory
% #2 refname of theory
{TM:Equilibrium}
% #3 label
{Equilibrium}
% #4 equation
{
\( \sum{F_x} = 0, \sum{F_y} = 0, \sum{M}=0 \)
}
% #5 description
{
\begin{symbDescription}
  \item{$\sum F_x$ is the sum of the x-components of all the forces (\si{\newton})}
  \item{$\sum F_y$ is the sum of the y-components of all the forces (\si{\newton})}
  \item{$M$ is the moment (\si{\newton\metre})}
\end{symbDescription}
}
% #6 Notes
{
  For a body in static equilibrium, the net forces and moments acting on the body will cancel out. 
  Assuming a 2D problem, the sum of the x-components of all the forces, $\sum F_x$, and the sum of the
  y-components of all the forces, $\sum F_y$, will be equal to 0. All forces and their distance from the 
  chosen point of rotation will create a net moment equal to 0.
}
% #7 Source
{
\citet{FredlundKrahn}
}
% #8 Referenced by
{
?.
}
% #9 Preconditions
{
\begin{itemize}
  \item \hyperref[A:HalfPlane]{A:HalfPlane}
\end{itemize}
}
% #1 derivation - not applicable by default
{}

%%%%%%%%%%%%%%%%%%%%%%%%%%%%%%%%%%%%%%

\deftheory
% #2 refname of theory
{GD:Weight}
% #3 label
{Weight}
% #4 equation
{
\( W = V\gamma \)
}
% #5 description
{
\begin{symbDescription}
  \item{$W$ is the weight (\si{\newton})}
  \item{$V$ is the volume (\si{\metre\cubed})}
  \item{$\gamma$ is the specific weight (\si{\frac{\newton}{\metre\cubed}})}
\end{symbDescription}
}
% #6 Notes
{
  None.
}
% #7 Source
{
\citet{WikipediaWeight}
}
% #8 Referenced by
{
?.
}
% #9 Preconditions
{
\begin{itemize}
\item add preconditions
\end{itemize}
}
% #1 derivation - not applicable by default
{}

%%%%%%%%%%%%%%%%%%%%%%%%%%%%%%%%%%%%%%

\deftheory
% #2 refname of theory
{GD:hsPressure}
% #3 label
{Hydrostatic Pressure}
% #4 equation
{
\( p = \gamma h \)
}
% #5 description
{
\begin{symbDescription}
  \item{$p$ is the pressure (\si{\pascal})}
  \item{$\gamma$ is the specific weight (\si{\frac{\newton}{\metre\cubed}})}
  \item{$h$ is the height (\si{\metre})}
\end{symbDescription}
}
% #6 Notes
{
  This equation is derived from Bernoulli's equation for a slow moving fluid through a porous material.
}
% #7 Source
{
\citet{WikipediaPressure}
}
% #8 Referenced by
{
?.
}
% #9 Preconditions
{
\begin{itemize}
\item add preconditions
\end{itemize}
}
% #1 derivation - not applicable by default
{}

%%%%%%%%%%%%%%%%%%%%%%%%%%%%%%%%%%%%%%

\bigskip

``Ref.\ By'' is used repeatedly with the different types of information. This
stands for Referenced By.  It means that the models, definitions and assumptions
listed reference the current model, definition or assumption. This information
is given for traceability.  Ref. By provides a pointer in the opposite direction
to what we commonly do.  You still need to have a reference in the other
direction pointing to the current model, definition or assumption.  As an
example, if T1 is referenced by G2, that means that G2 will explicitly include a
reference to T1.

~\newline

\subsubsection{Refined Theories (RT)}\label{sec_gendef}

\emph{TODO}

Refined theories are a refinement of one or more other theories. The refined theories are less abstract than the background theories.  Generally the reduction in
abstraction is possible through invoking (using/referencing) Assumptions. For
instance, the RT could be Newton's Law of Cooling stated abstracting.

This section collects the laws and equations that will be used in building the
final theories.

\subsubsection{Data Definitions}\label{sec_datadef}

\emph{TODO}

The Data Definitions are definitions of symbols and equations that are given for
the problem.  They are not derived; they are simply used by other models.

All Data Definitions should be used (referenced) by at least one other model.


\subsubsection{Data Types}\label{sec_datatypes}

\emph{TODO}

This section is optional.  In many scientific computing programs it isn't
necessary, since the inputs and outpus are straightforward types, like reals,
integers, and sequences of reals and integers.  However, for some problems it is
very helpful to capture the type information.

The data types are not derived; they are simply stated and used by other models.

All data types must be used by at least one of the models.

For the mathematical notation for expressing types, the recommendation is to use
the notation of~\citet{HoffmanAndStrooper1995}.

This section collects and defines all the data types needed to document the
models. Modify the examples below for your problem, and add additional
definitions as appropriate.

~\newline

\noindent
\begin{minipage}{\textwidth}
\renewcommand*{\arraystretch}{1.5}
\begin{tabular}{| p{\colAwidth} | p{\colBwidth}|}
  \hline
  \rowcolor[gray]{0.9}
  Type Name & Name for Type\\
  \hline
  Type Def & mathematical definition of the type\\
  \hline
  Description & description here
  \\
  \hline
  Sources & Citation here, if the type is borrowed from another source\\
  \hline
\end{tabular}
\end{minipage}\\

\subsubsection{Final Theories} \label{sec_instance}    

The motivation for this section is to reduce the problem defined in ``Physical
System Description'' (Section~\ref{sec_phySystDescrip}) to one expressed in
mathematical terms. The FTs are built by refining the RTs and BTs.  This
section should remain abstract.  The SRS should specify the requirements without
considering the implementation.

This section transforms the problem defined in Section~\ref{Sec_pd} into 
one which is expressed in mathematical terms. It uses concrete symbols defined 
in Section~\ref{sec_datadef} to replace the abstract symbols in the models 
identified in Sections~\ref{sec_theoretical} and~\ref{sec_gendef}.

The goals are solved by referencing your final theories.

\subsubsection{Input Data Constraints} \label{sec_DataConstraints}    

Table~\ref{TblInputVar} shows the data constraints on the input output
variables.  The column for physical constraints gives the physical limitations
on the range of values that can be taken by the variable.  The column for
software constraints restricts the range of inputs to reasonable values.  The
software constraints will be helpful in the design stage for picking suitable
algorithms.  The constraints are conservative, to give the user of the model the
flexibility to experiment with unusual situations.  The column of typical values
is intended to provide a feel for a common scenario.  The uncertainty column
provides an estimate of the confidence with which the physical quantities can be
measured.  This information would be part of the input if one were performing an
uncertainty quantification exercise.

The specification parameters in Table~\ref{TblInputVar} are listed in
Table~\ref{TblSpecParams}.

\begin{table}[!h]
  \caption{Input Variables} \label{TblInputVar}
  \renewcommand{\arraystretch}{1.2}
\noindent \begin{longtable*}{l l l l c} 
  \toprule
  \textbf{Var} & \textbf{Physical Constraints} & \textbf{Software Constraints} &
                             \textbf{Typical Value} & \textbf{Uncertainty}\\
  \midrule 
  $L$ & $L > 0$ & $L_{\text{min}} \leq L \leq L_{\text{max}}$ & 1.5 \si[per-mode=symbol] {\metre} & 10\%
  \\
  \bottomrule
\end{longtable*}
\end{table}

\noindent 
\begin{description}
\item[(*)] you might need to add some notes or clarifications
\end{description}

\begin{table}[!h]
\caption{Specification Parameter Values} \label{TblSpecParams}
\renewcommand{\arraystretch}{1.2}
\noindent \begin{longtable*}{l l} 
  \toprule
  \textbf{Var} & \textbf{Value} \\
  \midrule 
  $L_\text{min}$ & 0.1 \si{\metre}\\
  \bottomrule
\end{longtable*}
\end{table}

\subsubsection{Properties of a Correct Solution} \label{sec_CorrectSolution}

A correct solution must exhibit fill in the details.  These properties are in
addition to the stated requirements.  There is no need to repeat the
requirements here.  These additional properties may not exist for every problem.
Examples include conservation laws (like conservation of energy or mass) and
known constraints on outputs, which are usually summarized in tabular form.  A
sample table is shown in Table~\ref{TblOutputVar}

\begin{table}[!h]
\caption{Output Variables} \label{TblOutputVar}
\renewcommand{\arraystretch}{1.2}
\noindent \begin{longtable*}{l l} 
  \toprule
  \textbf{Var} & \textbf{Physical Constraints} \\
  \midrule 
  $T_W$ & $T_\text{init} \leq T_W \leq T_C$ (by~\aref{A_charge})
  \\
  \bottomrule
\end{longtable*}
\end{table}

This section is not for test cases or techniques for verification and
validation.  Those topics will be addressed in the Verification and Validation
plan.

\section{Requirements}

The requirements refine the goal statement.  They will make heavy use of
references to the instance models.

This section provides the functional requirements, the business tasks that the
software is expected to complete, and the nonfunctional requirements, the
qualities that the software is expected to exhibit.

\subsection{Functional Requirements}

\noindent \begin{itemize}

\item[R\refstepcounter{reqnum}\thereqnum \label{R_Inputs}:] Requirements
    for the inputs that are supplied by the user.  This information has to be
    explicit.

\item[R\refstepcounter{reqnum}\thereqnum \label{R_OutputInputs}:] It isn't
    always required, but often echoing the inputs as part of the output is a
    good idea.

\item[R\refstepcounter{reqnum}\thereqnum \label{R_Calculate}:] Calculation
    related requirements.

\item[R\refstepcounter{reqnum}\thereqnum \label{R_VerifyOutput}:]
  Verification related requirements.

\item[R\refstepcounter{reqnum}\thereqnum \label{R_Output}:] Output related
    requirements.

\end{itemize}

Every IM should map to at least one requirement, but not every requirement has
to map to a corresponding IM.

\subsection{Nonfunctional Requirements}

List your nonfunctional requirements.  You may consider using a fit criterion to
make them verifiable. The goal is for the nonfunctional requirements to be
unambiguous, abstract and verifiable.  This isn't easy to show succinctly, so a
good strategy may be to give a ``high level'' view of the requirement, but allow
for the details to be covered in the Verification and Validation document. An
absolute requirement on a quality of the system is rarely needed.  For instance,
an accuracy of 0.0101 \% is likely fine, even if the requirement is for 0.01 \%
accuracy.  Therefore, the emphasis will often be more on describing now well the
quality is achieved, through experimentation, and possibly theory, rather than
meeting some bar that was defined a priori. You do not need an entry for
correctness in your NFRs.  The purpose of the SRS is to record the requirements
that need to be satisfied for correctness. Any statement of correctness would
just be redundant. Rather than discuss correctness, you can characterize how far
away from the correct (true) solution you are allowed to be.  This is discussed
under accuracy.

\noindent \begin{itemize}

\item[NFR\refstepcounter{nfrnum}\thenfrnum \label{NFR_Accuracy}:]
  \textbf{Accuracy} Characterize the accuracy by giving the context/use for
    the software.  Maybe something like, ``The accuracy of the computed
    solutions should meet the level needed for $<$engineering or scientific
    application$>$.  The level of accuracy achieved by \emph{VDisp}{} shall be
    described following the procedure given in Section~X of the Verification and
    Validation Plan.''  A link to the VnV plan would be a nice extra.

\item[NFR\refstepcounter{nfrnum}\thenfrnum \label{NFR_Usability}:] \textbf{Usability}
  Characterize the usability by giving the context/use for the software.
    You should likely reference the user characteristics section.  The level of
    usability achieved by the software shall be described following the
    procedure given in Section~X of the Verification and Validation Plan.  A
    link to the VnV plan would be a nice extra.

\item[NFR\refstepcounter{nfrnum}\thenfrnum \label{NFR_Maintainability}:]
  \textbf{Maintainability} The effort required to make any of the likely
    changes listed for \emph{VDisp}{} should be less than FRACTION of the original
    development time.  FRACTION is then a symbolic constant that can be defined
    at the end of the report.

\item[NFR\refstepcounter{nfrnum}\thenfrnum \label{NFR_Portability}:]
  \textbf{Portability} This NFR is easier to write than the others.  The
    systems that \emph{VDisp}{} should run on should be listed here.  When possible
    the specific versions of the potential operating environments should be
    given.  To make the NFR verifiable a statement could be made that the tests
    from a given section of the VnV plan can be successfully run on all of the
    possible operating environments.

\item Other NFRs that might be discussed include verifiability,
  understandability and reusability.

\end{itemize}

\section{Likely Changes}    

\noindent \begin{itemize}

\item[LC\refstepcounter{lcnum}\thelcnum\label{LC_meaningfulLabel}:] Give the
    likely changes, with a reference to the related assumption (aref), as
    appropriate.

\end{itemize}

\section{Unlikely Changes}    

\noindent \begin{itemize}

\item[LC\refstepcounter{lcnum}\thelcnum\label{LC_meaningfulLabel}:] Give the
    unlikely changes.  The design can assume that the changes listed will not
    occur.

\end{itemize}

\section{Traceability Matrices and Graphs}

The purpose of the traceability matrices is to provide easy references on what
has to be additionally modified if a certain component is changed.  Every time a
component is changed, the items in the column of that component that are marked
with an ``X'' may have to be modified as well.  Table~\ref{Table:trace} shows
the dependencies of theoretical models, general definitions, data definitions,
and instance models with each other. Table~\ref{Table:R_trace} shows the
dependencies of instance models, requirements, and data constraints on each
other. Table~\ref{Table:A_trace} shows the dependencies of theoretical models,
general definitions, data definitions, instance models, and likely changes on
the assumptions.

You will have to modify these tables for your problem.

The traceability matrix is not generally symmetric.  If GD1 uses A1, that means
that GD1's derivation or presentation requires invocation of A1.  A1 does not
use GD1.  A1 is ``used by'' GD1.

The traceability matrix is challenging to maintain manually.  Please do your
best.  In the future tools (like Drasil) will make this much easier.

\afterpage{
\begin{landscape}
\begin{table}[h!]
\centering
\begin{tabular}{|c|c|c|c|c|c|c|c|c|c|c|c|c|c|c|c|c|c|c|c|}
\hline
	& \aref{A_OnlyThermalEnergy}& \aref{A_hcoeff}& \aref{A_mixed}& \aref{A_tpcm}& \aref{A_const_density}& \aref{A_const_C}& \aref{A_Newt_coil}& \aref{A_tcoil}& \aref{A_tlcoil}& \aref{A_Newt_pcm}& \aref{A_charge}& \aref{A_InitTemp}& \aref{A_OpRangePCM}& \aref{A_OpRange}& \aref{A_htank}& \aref{A_int_heat}& \aref{A_vpcm}& \aref{A_PCM_state}& \aref{A_Pressure} \\
\hline
\tref{T_COE}        & X& & & & & & & & & & & & & & & & & & \\ \hline
\tref{T_SHE}        & & & & & & & & & & & & & & & & & & & \\ \hline
\tref{T_LHE}        & & & & & & & & & & & & & & & & & & & \\ \hline
\dref{NL}           & & X& & & & & & & & & & & & & & & & & \\ \hline
\dref{ROCT}         & & & X& X& X& X& & & & & & & & & & & & & \\ \hline
\ddref{FluxCoil}    & & & & & & & X& X& X& & & & & & & & & & \\ \hline
\ddref{FluxPCM}     & & & X& X& & & & & & X& & & & & & & & & \\ \hline
\ddref{D_HOF}       & & & & & & & & & & & & & & & & & & & \\ \hline
\ddref{D_MF}        & & & & & & & & & & & & & & & & & & & \\ \hline
\iref{ewat}         & & & & & & & & & & & X& X& & X& X& X& & & X \\ \hline
\iref{epcm}         & & & & & & & & & & & & X& X& & & X& X& X& \\ \hline
\iref{I_HWAT}       & & & & & & & & & & & & & & X& & & & & X \\ \hline
\iref{I_HPCM}       & & & & & & & & & & & & & X& & & & & X & \\ \hline
\lcref{LC_tpcm}     & & & & X& & & & & & & & & & & & & & & \\ \hline
\lcref{LC_tcoil}    & & & & & & & & X& & & & & & & & & & & \\ \hline
\lcref{LC_tlcoil}   & & & & & & & & & X& & & & & & & & & & \\ \hline
\lcref{LC_charge}   & & & & & & & & & & & X& & & & & & & & \\ \hline
\lcref{LC_InitTemp} & & & & & & & & & & & & X& & & & & & & \\ \hline
\lcref{LC_htank}    & & & & & & & & & & & & & & & X& & & & \\
\hline
\end{tabular}
\caption{Traceability Matrix Showing the Connections Between Assumptions and Other Items}
\label{Table:A_trace}
\end{table}
\end{landscape}
}

\begin{table}[h!]
\centering
\begin{tabular}{|c|c|c|c|c|c|c|c|c|c|c|c|c|c|c|c|c|c|c|c|c|c|c|c|}
\hline        
	& \tref{T_COE}& \tref{T_SHE}& \tref{T_LHE}& \dref{NL}& \dref{ROCT} & \ddref{FluxCoil}& \ddref{FluxPCM} & \ddref{D_HOF}& \ddref{D_MF}& \iref{ewat}& \iref{epcm}& \iref{I_HWAT}& \iref{I_HPCM} \\
\hline
\tref{T_COE}     & & & & & & & & & & & & & \\ \hline
\tref{T_SHE}     & & & X& & & & & & & & & & \\ \hline
\tref{T_LHE}     & & & & & & & & & & & & & \\ \hline
\dref{NL}        & & & & & & & & & & & & & \\ \hline
\dref{ROCT}      & X& & & & & & & & & & & & \\ \hline
\ddref{FluxCoil} & & & & X& & & & & & & & & \\ \hline
\ddref{FluxPCM}  & & & & X& & & & & & & & & \\ \hline
\ddref{D_HOF}    & & & & & & & & & & & & & \\ \hline
\ddref{D_MF}     & & & & & & & & X& & & & & \\ \hline
\iref{ewat}      & & & & & X& X& X& & & & X& & \\ \hline
\iref{epcm}      & & & & & X& & X& & X& X& & & X \\ \hline
\iref{I_HWAT}    & & X& & & & & & & & & & & \\ \hline
\iref{I_HPCM}    & & X& X& & & & X& X& X& & X& & \\
\hline
\end{tabular}
\caption{Traceability Matrix Showing the Connections Between Items of Different Sections}
\label{Table:trace}
\end{table}

\begin{table}[h!]
\centering
\begin{tabular}{|c|c|c|c|c|c|c|c|}
\hline
	& \iref{ewat}& \iref{epcm}& \iref{I_HWAT}& \iref{I_HPCM}& \ref{sec_DataConstraints}& \rref{R_RawInputs}& \rref{R_MassInputs} \\
\hline
\iref{ewat}            & & X& & & & X& X \\ \hline
\iref{epcm}            & X& & & X& & X& X \\ \hline
\iref{I_HWAT}          & & & & & & X& X \\ \hline
\iref{I_HPCM}          & & X& & & & X& X \\ \hline
\rref{R_RawInputs}     & & & & & & & \\ \hline
\rref{R_MassInputs}    & & & & & & X& \\ \hline
\rref{R_CheckInputs}   & & & & & X& & \\ \hline
\rref{R_OutputInputs}  & X& X& & & & X& X \\ \hline
\rref{R_TempWater}     & X& & & & & & \\ \hline 
\rref{R_TempPCM}       & & X& & & & & \\ \hline
\rref{R_EnergyWater}   & & & X& & & & \\ \hline
\rref{R_EnergyPCM}     & & & & X& & & \\ \hline
\rref{R_VerifyOutput}  & & & X& X& & & \\ \hline
\rref{R_timeMeltBegin} & & X& & & & & \\ \hline
\rref{R_timeMeltEnd}   & & X& & & & & \\ 
\hline
\end{tabular}
\caption{Traceability Matrix Showing the Connections Between Requirements and Instance Models}
\label{Table:R_trace}
\end{table}

The purpose of the traceability graphs is also to provide easy references on
what has to be additionally modified if a certain component is changed.  The
arrows in the graphs represent dependencies. The component at the tail of an
arrow is depended on by the component at the head of that arrow. Therefore, if a
component is changed, the components that it points to should also be
changed. Figure~\ref{Fig_ATrace} shows the dependencies of theoretical models,
general definitions, data definitions, instance models, likely changes, and
assumptions on each other. Figure~\ref{Fig_RTrace} shows the dependencies of
instance models, requirements, and data constraints on each other.

% \begin{figure}[h!]
% 	\begin{center}
% 		%\rotatebox{-90}
% 		{
% 			\includegraphics[width=\textwidth]{ATrace.png}
% 		}
% 		\caption{\label{Fig_ATrace} Traceability Matrix Showing the Connections Between Items of Different Sections}
% 	\end{center}
% \end{figure}


% \begin{figure}[h!]
% 	\begin{center}
% 		%\rotatebox{-90}
% 		{
% 			\includegraphics[width=0.7\textwidth]{RTrace.png}
% 		}
% 		\caption{\label{Fig_RTrace} Traceability Matrix Showing the Connections Between Requirements, Instance Models, and Data Constraints}
% 	\end{center}
% \end{figure}

\section{Development Plan}

This section is optional.  It is used to explain the plan for developing the
software.  In particular, this section gives a list of the order in which the
requirements will be implemented.  In the context of a course like CAS 741, this
is where you can indicate which requirements will be implemented as part of the
course, and which will be ``faked'' as future work.  This section can be
organized as a prioritized list of requirements, or it could should the
requirements that will be implemented for ``phase 1'', ``phase 2'', etc.

\section{Values of Auxiliary Constants}

Show the values of the symbolic parameters introduced in the report.

The definition of the requirements will likely call for SYMBOLIC\_CONSTANTS.
Their values are defined in this section for easy maintenance.

The value of FRACTION, for the Maintainability NFR would be given here.

\newpage

\noindent The following is not part of the template, just some things to consider
  when filing in the template.

\noindent Grammar, flow and \LaTeX advice:
\begin{itemize}
\item For Mac users \texttt{*.DS\_Store} should be in \texttt{.gitignore}
\item \LaTeX{} and formatting rules
\begin{itemize}
\item Variables are italic, everything else not, includes subscripts (link to
  document)
\begin{itemize}
\item \href{https://physics.nist.gov/cuu/pdf/typefaces.pdf}{Conventions}
\item Watch out for implied multiplication
\end{itemize}
\item Use BibTeX
\item Use cross-referencing
\end{itemize}
\item Grammar and writing rules
\begin{itemize}
\item Acronyms expanded on first usage (not just in table of acronyms)
\item ``In order to'' should be ``to''
\end{itemize}
\end{itemize}

\noindent Advice on using the template:
\begin{itemize}
\item Difference between physical and software constraints
\item Properties of a correct solution means \emph{additional} properties, not
  a restating of the requirements (may be ``not applicable'' for your problem).
  If you have a table of output constraints, then these are properties of a
  correct solution.
\item Assumptions have to be invoked somewhere
\item ``Referenced by'' implies that there is an explicit reference
\item Think of traceability matrix, list of assumption invocations and list of
  reference by fields as automatically generatable
\item If you say the format of the output (plot, table etc), then your
  requirement could be more abstract
\end{itemize}

\newpage

\bibliographystyle {plainnat}
\bibliography {References}

\end{document}
