\documentclass[11pt,fleqn]{article}

\usepackage{blindtext}

\newcommand{\indentpar}{\phantom{=}}
\newcommand{\bli}{\begin{itemize}}
\newcommand{\eli}{\end{itemize}}

\begin{document}

    %% Title Page %%%%%%%%%%%%%%%%%%%%%%%%%%%%%%%%%%%%%%%%%%%%%%%%%
      
      \title{Inconsistencies in Settlement Analysis Textbook and FORTRAN Code}
      \date{\today}
      \author{Emil Soleymani, Dr.~Spencer Smith\\ McMaster University}
      \maketitle

      \medskip

      % Description
      \indentpar During review of the legacy FORTRAN code for the VDispl software, 
      we referenced the Settlements Analysis book linked in the README 
      file to correlate each calculation with its corresponding 
      theories and formulas. This process revealed many potential errors 
      in the code and the formulas presented in the textbook, along with 
      some inconsistencies between calculations made in the code and 
      calculations shown in the textbook. This document is a compilation 
      of all such occurrences.
      
    %%%%%%%%%%%%%%%%%%%%%%%%%%%%%%%%%%%%%%%%%%%%%%%%%%%%%%%%%%%%%%%
    
    \pagebreak
    
    %% Notes %%%%%%%%%%%%%%%%%%%%%%%%%%%%%%%%%%%%%%%%%%%%%%%%%%%%%

    \section*{Notes}
    \indentpar All references to FORTRAN code are relative to line numbers 
    in the \emph{most recent} version of the code, found under 
    \textbf{Resources/fortran\_src/vdisp.for} while all references to the pages
    in the Settlements Analysis textbook refer to a page number of the PDF linked 
    in the README.md file.

    %%%%%%%%%%%%%%%%%%%%%%%%%%%%%%%%%%%%%%%%%%%%%%%%%%%%%%%%%%%%%%%
    
    \medskip

    %% Errors in Textbook %%%%%%%%%%%%%%%%%%%%%%%%%%%%%%%%%%%%%%%%%%
    \section*{Potential Errors in Textbook}

    \indentpar Having examined the many formulas in the book during the research and 
    analysis of the FORTRAN code, we have come across some potential errors in the
    textbook, ranging from variable names to questionable units.
   
    \subsection*{Units for unit weight of water, $\gamma_w$:} 
    \indentpar On page 12 of the Settlements Analysis PDF, 
    under heading (3), there is a reference to a constant $\gamma_w$, 
    which represents the unit weight of water. This constant is known
    to be 9.807 $kN/m^3$ in SI units, however the textbook lists it as 
    0.031tsf. The unit tsf — tonnes per square foot — seems to be a unit
    of pressure which can be converted to $N/m^2$. We have no current 
    explanation for this discrepancy.

    \subsection*{Leonard's and Frost Model:}
    \indentpar On page 169 of the Settlements Analysis PDF, 
    the equation (F-1) has a variable $c_i$ which represents 
    the correction to account for strain relief from embedment,
    but in subsequent explanations and in the FORTRAN code it 
    is labelled as $c_1$. \\
    \indentpar Furthermore, the equation (F-4) on the 
    same page is as follows:
    \begin{center}
        $\phi_{ax} = \phi_{ps} + [\frac{\phi_{ps} - 32}{3}]$
    \end{center}
    \indentpar However, in the 1994 publication of the textbook, the equation
    is slightly altered to:
    \begin{center}
        $\phi_{ax} = \phi_{px} + [\frac{\phi_{ps} - 32}{3}]$
    \end{center}
    \indentpar The variable $\phi_{px}$ is not defined nor mentioned in the textbook. 
    The FORTRAN code uses the 1990 version of this formula. It is not clear
    whether the 1994 version introduced a typo or a meaningful change. \\
    \indentpar Finally, equation (F-5) on page 170 of the Settlements 
    Analysis PDF has a variable called $K_p$. The textbook never describes 
    nor mentions this variable. It does however mention a variable $K_d$, and also
    provides a formula for it. It seems equation (F-5) has a typo in the subscript.

    \subsection*{Output File Explanation:}
    \indentpar In Appendix F, Table F-3 (p. 167) of the Settlements Analysis Textbook,
    there is a typo in the explanation of the logic for deciding to output “Equilibrium 
    saturated above water table” or “Equilibrium hydrostatic profile above water table“.
    On line 13, the conditional statement checks for NOPT$=0$ instead of NOPT$=1$. The 
    truth table should match the following, where $f=1$ means the program outputs “Equilibrium 
    saturated above water table”, else program outputs “Equilibrium hydrostatic profile above 
    water table“:
    \begin{center}
        \begin{tabular}{|c|c|c|}
            \hline 
            IOPTION & NOPT & $f$ \\
            \hline 
            0 & 0 & 1 \\
            \hline
            0 & 1 & 1 \\
            \hline 
            1 & 0 & 0 \\
            \hline 
            1 & 1 & 1 \\
            \hline
        \end{tabular}
        \\
        \medskip
        \textit{Note}: $f \equiv$ IOPTION$\Rightarrow$NOPT.
    \end{center}

    \subsection*{Input File Examples:}
    \indentpar In Appendix F of the Settlements Analysis Textbook, there are multiple example 
    input files, and their corresponding output. Table F-6 on p.176, titled \emph{Settlment of 
    Granular Soil, Schmertmann Model}, has a typo. In the last line of Table F-6a, there are 
    two entries: “3 10.00”. This line should only have one entry which represents \emph{time in 
    years after construction}. By examining the output file, Table F-6b, it is clear that the value
    of \emph{time in years after construction} should be 10, thus the \emph{3} at the beginning of 
    the line should be removed. Upon removal the output of the \emph{vdisp.for} program matches
    Table F-6b.
    %%%%%%%%%%%%%%%%%%%%%%%%%%%%%%%%%%%%%%%%%%%%%%%%%%%%%%%%%%%%%%%   
     
\end{document}